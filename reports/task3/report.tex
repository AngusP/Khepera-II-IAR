
%     %%%%%%%%%%%%%%%
%
%     P A C K A G E S
%
%     %%%%%%%%%%%%%%%

\documentclass[11pt, a4paper]{article}
\usepackage{fontspec}
\usepackage{caption}
\usepackage{mathtools}
\usepackage{gensymb}
\usepackage{float}

% DOCUMENT LAYOUT
\usepackage{geometry}
\geometry{a4paper, textwidth=42em, textheight=70em, marginparsep=0.5em, marginparwidth=3.5em}
\setlength\parindent{0em}
\setlength\parskip{0.75em}
\captionsetup{width=0.8\textwidth}

% FONTS
\usepackage[usenames,dvipsnames]{xcolor}
\usepackage{xunicode}
\usepackage{xltxtra}
\defaultfontfeatures{Mapping=tex-text}
%\setromanfont [Ligatures={Common}, Numbers={OldStyle}, Variant=01]{Linux Libertine O}
%\setmonofont[Scale=0.8]{Monaco}
%%% modified by Karol Kozioł for ShareLaTeX use
\setmainfont[
  Ligatures={Common}, Numbers={OldStyle}, Variant=01,
  BoldFont=LinLibertine_RB.otf,
  ItalicFont=LinLibertine_RI.otf,
  BoldItalicFont=LinLibertine_RBI.otf
]{LinLibertine_R.otf}
\setmonofont[Scale=0.8]{DejaVuSansMono.ttf}

% HEADINGS
\usepackage{sectsty}
\usepackage[normalem]{ulem}
\sectionfont{\mdseries\upshape\Large}
\subsectionfont{\mdseries\scshape\normalsize}
\subsubsectionfont{\mdseries\upshape\normalsize}

\renewenvironment{abstract}{%
{\mdseries\scshape\Large\abstractname}
\vspace{1em}\\
}{\par\noindent}


\usepackage[superscript]{cite}

% LISTINGS
\usepackage{listings}
\usepackage{color}
\usepackage{appendix}

\usepackage{color}
\definecolor{codered}{rgb}{0.61,0.21,0.18}
\definecolor{codegreen}{rgb}{0,0.6,0}
\definecolor{codegray}{rgb}{0.5,0.5,0.5}
\definecolor{codepurple}{rgb}{0.58,0,0.82}
\definecolor{backcolour}{rgb}{1.0,1.0,1.0}
\lstset{
  backgroundcolor=\color{backcolour},   
  commentstyle=\color{codegray},
  keywordstyle=\color{codered},
  numberstyle=\tiny\color{codegreen},
  stringstyle=\color{codepurple},
  basicstyle=\footnotesize\ttfamily,        % the size of the fonts that are used for the code
  breaklines=true,                          % sets automatic line breaking
  keepspaces=true,                          % keeps spaces in text, useful for keeping indentation of code
  showspaces=false,                         % show spaces everywhere adding particular underscores; it overrides 'showstringspaces'
  showstringspaces=false,                   % underline spaces within strings only
  showtabs=false,                           % show tabs within strings adding particular underscores
  stepnumber=2,                             % the step between two line-numbers. If it's 1, each line will be numbered
  tabsize=4, 	                            % sets default tabsize to 2 spaces
  title=\lstname                            % show the filename of files included with \lstinputlisting
}


%     %%%%%%%%%%%%%%%
%
%     D O C U M E N T
%
%     %%%%%%%%%%%%%%%


\begin{document}
\title{IAR Task 3 Report}
\author{Angus Pearson -- s1311631\\ Jevgenij Zubovskij -- s1346981}
\date{\today}
\maketitle

%       ^v^v^v^v^v^v^v^v^v^v^v^v^v^v^v^v^v^v^v^v^v^v^v^v^v^v^v^v^v^v^v^v^v^v^v^


\begin{abstract}
 

TODO


\end{abstract}

%       ^v^v^v^v^v^v^v^v^v^v^v^v^v^v^v^v^v^v^v^v^v^v^v^v^v^v^v^v^v^v^v^v^v^v^v^


\section{Introduction}
\label{Introduction}

TODO


%       ^v^v^v^v^v^v^v^v^v^v^v^v^v^v^v^v^v^v^v^v^v^v^v^v^v^v^v^v^v^v^v^v^v^v^v^

\newpage
\section{Path Planning}
\label{Pathing}

One of the two primary goals is to plan a path for the robot that would allow it to collect the maximum amount of food during a given time period. 

\subsubsection{Planning Space}

\subsubsection{Method Chosen}

The two approaches considered when deciding how to configure the planning space were:

\begin{itemize}
	\item \textit{Occupancy Grid}
	\item \textit{Graph (Topological Map)}
\end{itemize}



Now, the difference between these two from the perspective of planning is the efficiency of algorithm working on the data and the pre-processing of the data going to the algorithm. Because an occupancy grid is stored on a \textit{Redis-Server} and it has very fine definition, the number of grid squares (referred to as \textit{Grid Cerlls}) is immense. Therefore, to obtain a graph representation from that, every time a path replanning would have to take place, the conversion would have to take place on a very significant number of points. Therefore, the occupancy grid was chosen due to no conversion needing to be made on the data during runtime for planning, but rather perhaps a small alteration for the planning algorithm.

\subsubsection{Alterations}

It was also decided that the \textit{Planning Grid} would have a granularity equal to or higher than the one of the occupancy grid. The reasons for such a decision are as follows:

\begin{itemize}

	\item Robot Dimensions     - the robot's dimensions are larger than the mapping granularity
	\item Odometry Distortion  - extremely finely planned path on an occupancy grid introduces many turns, increasing odometry drift rate considerably \cite{task2_report}
	\item Runtime Optimization - as long as there is a path to the goal and the robot can follow it, we wish to save as much processing time as we humanely can and plannign with bigger cell dimensions helps us save calculation time

\end{itemize}

\subsection{Planning Algorithm}


\subsubsection{Research}

Firstly, there are many options for the \textit{Path-Planning Algorithm}

\begin{itemize}

	\item \textit{Floyd-Warshall Algorithm}\cite{path_warshall} - a many-to-many algorithm for planning routes
	\item \textit{Djikstra Algorithm}\cite{path_warshall} - a one-to-many algorithm
	\item \textit{A* Algorithm}\cite{path_astar}	- a one-to-one algorithm

These are some of the most common and roust pathign algorihtms used in pathing. However, even based on the iformation above it can be concluded that Floyd-Warshall algorithm is not waht we were lookign because w have a single strating point - the position of the robot. Moreover, empirical data\cite{path_efficiency} shows that the best-first search algorithm is considerably faster than the Djikstra algorithm and in fact is in general considered the fastest planning algorithm and it's varioations as well as its purest form are used industry-wide in robotics. More reasons for choosing A* can be outlined in \textit{\S\ref{Alterations_Algorithm}}

\subsubsection{Method Chosen} 

Therefore, A* was chosen to be the algorithm of choice for path calculation to the target grid cell. The pseudocode explaining its basic operation is as follows:

\begin{figure}
function A*(start, goal)
    // The set of nodes already evaluated.
    closedSet := {}
    // The set of currently discovered nodes still to be evaluated.
    // Initially, only the start node is known.
    openSet := {start}
    // For each node, which node it can most efficiently be reached from.
    // If a node can be reached from many nodes, cameFrom will eventually contain the
    // most efficient previous step.
    cameFrom := the empty map

    // For each node, the cost of getting from the start node to that node.
    gScore := map with default value of Infinity
    // The cost of going from start to start is zero.
    gScore[start] := 0 
    // For each node, the total cost of getting from the start node to the goal
    // by passing by that node. That value is partly known, partly heuristic.
    fScore := map with default value of Infinity
    // For the first node, that value is completely heuristic.
    fScore[start] := heuristic_cost_estimate(start, goal)

    while openSet is not empty
        current := the node in openSet having the lowest fScore[] value
        if current = goal
            return reconstruct_path(cameFrom, current)

        openSet.Remove(current)
        closedSet.Add(current)
        for each neighbor of current
            if neighbor in closedSet
                continue		// Ignore the neighbor which is already evaluated.
            // The distance from start to a neighbor
            tentative_gScore := gScore[current] + dist_between(current, neighbor)
            if neighbor not in openSet	// Discover a new node
                openSet.Add(neighbor)
            else if tentative_gScore >= gScore[neighbor]
                continue		// This is not a better path.

            // This path is the best until now. Record it!
            cameFrom[neighbor] := current
            gScore[neighbor] := tentative_gScore
            fScore[neighbor] := gScore[neighbor] + heuristic_cost_estimate(neighbor, goal)

    return failure

function reconstruct_path(cameFrom, current)
    total_path := [current]
    while current in cameFrom.Keys:
        current := cameFrom[current]
        total_path.append(current)
    return total_path
\end{figure} 

\subsubsection{Alterations}
\label{Alterations_Algorithm}

Moreover, because the testing space is quite sparsely populated and is relatively small, it ws decided to save even more computational time by determining the closest (and thus the one we wish to calculate an exact path to) using simple \textit{Euclidian Distances}.


\end{itemize}

\subsubsection{Method Chosen}
\subsubsection{Alterations}




\subsubsection{Integration}


\subsubsection{Research}
\subsubsection{Method Chosen}
\subsubsection{Alterations}



%       ^v^v^v^v^v^v^v^v^v^v^v^v^v^v^v^v^v^v^v^v^v^v^v^v^v^v^v^v^v^v^v^v^v^v^v^

\newpage
\begin{thebibliography}{}

\bibitem{task1_report} 
\par{IAR 2016 Task1 Report} \\
\textit{Angus Pearson, Jevgenij Zubovskij}

\bibitem{task2_report} 
\par{IAR 2016 Task1 Report} \\
\textit{Angus Pearson, Jevgenij Zubovskij}


\bibitem{path_warshall} 
\par{Theory of Algorithms (lecture)}
\textit{http://cs.winona.edu/lin/cs440/ch08-2.pdf}

\bibitem{path_djikstra} 
\par{Greedy Algorithms (online article)}
\textit{http://www.geeksforgeeks.org/greedy-algorithms-set-6-dijkstras-shortest-path-algorithm/}

\bibitem{path_astar}
\par{A* Comparison (online article)}
\textit{http://theory.stanford.edu/~amitp/GameProgramming/AStarComparison.html}

\bibitem{path_efficiency} 
\par{Comparative Study of Path Planning Algorithms}
\textit{http://research.ijcaonline.org/volume39/number5/pxc3877058.pdf}


\bibitem{path_astar_pseudo} 
\par{A* Search Algorithm (online article)}
\textit{https://en.wikipedia.org/wiki/A*_search_algorithm}



\end{thebibliography}

\newpage
\begin{appendices}
\section*{Appendix}
\subsection{Code Listings}
\lstinputlisting[language=python]{../../main.py}
\lstinputlisting[language=python]{../../data.py}
\lstinputlisting[language=python]{../../comms.py}
\lstinputlisting[language=python]{../../state.py}
\lstinputlisting[language=python]{../../pathing_state.py}
\lstinputlisting[language=python]{../../pathing_algorithm.py}
\lstinputlisting[language=python]{../../astar.py}
\lstinputlisting[language=python]{../../navigation_state.py}
\lstinputlisting[language=python]{../../navigation_algorithm.py}
\lstinputlisting[language=python]{../../odometry_algorithm.py}
\lstinputlisting[language=python]{../../odometry_state.py}
\lstinputlisting[language=python]{../../constants.py}
\lstinputlisting{../../requirements.txt}
\end{appendices}


\end{document}
