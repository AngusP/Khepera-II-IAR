\documentclass[11pt,a4wide]{article}


%     %%%%%%%%%%%%%%%
%
%     P A C K A G E S
%
%     %%%%%%%%%%%%%%%

\usepackage{geometry}
\geometry{
  a4paper,
  total={210mm,297mm},
  left=20mm,
  right=20mm,
  top=20mm,
  bottom=20mm,
}



%     %%%%%%%%%%%%%%%
%
%     D O C U M E N T
%
%     %%%%%%%%%%%%%%%

\begin{document}

\title{IAR Task 1 Report}
\author{s1311631, s1346981}
\date{\today}
\maketitle

%       ^v^v^v^v^v^v^v^v^v^v^v^v^v^v^v^v^v^v^v^v^v^v^v^v^v^v^v^v^v^v^v^v^v^v^v^

\section{Introduction}

The first assignment entails utilising the Infra-red distance sensors on the Kephera 
robot, remotely controlled over serial. We have elected to use Python as the 
implementation language for this practical.

The task requires that the robot be able to autonomously move around it's environment,
``without hitting obstacles or getting stuck in corners or dead-ends. Second, it should 
tend to follow long walls, keeping a consistent distance away from the wall.''

Our approach was to investigate the efficacy of PID controlers to acheive 
these behaviours - With PID, maintaining a constant distance from a wall arises as
a behaviour naturally, given the controller will always be attempting to zero the 
error, which is a distance to the wall. Built ontop of PID are control methods to
allow the robot to navigate away from an obstacle.

\section{Exploring Different Methods of Control}

\subsection{PID Ease-Off Error Mitigation}

% Commit 5837a4b

The first iteration used the distance-derived errors from the forward-facing 
sensors only. When the error passes a set threshold, where the wall is within a 
fixed distance from the robot, we stop moving forwards and instead turn \emph{away} 
from the side that is close to the wall until the error has reduced to below the 
threshold.

This approach had some promising behaviours: The robot was able to prevent itself 
from hitting a small object directly in it's path; Similarly it was able to follow 
a straight wall to some degree. However, wall following would eiher oscillate 
towards the wall, bumping into it or depart after following for a short distance. 

% Commit f1258e7

We then augmented this method with the distance and PID error data from the other
forward-facing sensors, having the effect of widening the Robot's field of view.
This prevented the robot from colliding with the wall as the side-facing sensors 
are now involved in the navigation. Furthermore, the robot will turn in a curve 
when the wall is approaching but not very close, and will turn on the spot if the 
robot is stuck in a corner or is very close to an obstacle.

\subsection{Force-Vector Control}

As an alternative method we consider treating the Error vector produced by the
PID Controller for each distance sensor and treating them as the magnitudes of a 
force pushing the robot along the axis of said sensor. Summing these force vectors
gives us an `Intended Direction' vector, which is used to correct the current 
trajectory away from the obstacle.

Experimentation indicates this method does not perform as well as the PID ease-off 
method above, as it is prone to `wobbling' from noisy sensor data, and does not
exhibit wall-following behaviour as the error vectors push in direct opposition to 
the wall, resulting in an incident-reflection equivalence in the angle the robot
takes to the wall.

The method is, however, somewhat more elegant insofar as the complexity of behaviour 
that emerges from a simple ruleset.

\section{Physical Limitations and Possible Improvements}


\end{document}
