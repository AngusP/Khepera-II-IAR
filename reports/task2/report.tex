
%     %%%%%%%%%%%%%%%
%
%     P A C K A G E S
%
%     %%%%%%%%%%%%%%%

\documentclass[11pt, a4paper]{article}
\usepackage{fontspec}
\usepackage{caption}

% DOCUMENT LAYOUT
\usepackage{geometry}
\geometry{a4paper, textwidth=42em, textheight=70em, marginparsep=0.5em, marginparwidth=3.5em}
\setlength\parindent{0em}
\setlength\parskip{0.75em}
\captionsetup{width=0.8\textwidth}

% FONTS
\usepackage[usenames,dvipsnames]{xcolor}
\usepackage{xunicode}
\usepackage{xltxtra}
\defaultfontfeatures{Mapping=tex-text}
%\setromanfont [Ligatures={Common}, Numbers={OldStyle}, Variant=01]{Linux Libertine O}
%\setmonofont[Scale=0.8]{Monaco}
%%% modified by Karol Kozioł for ShareLaTeX use
\setmainfont[
  Ligatures={Common}, Numbers={OldStyle}, Variant=01,
  BoldFont=LinLibertine_RB.otf,
  ItalicFont=LinLibertine_RI.otf,
  BoldItalicFont=LinLibertine_RBI.otf
]{LinLibertine_R.otf}
\setmonofont[Scale=0.8]{DejaVuSansMono.ttf}

% HEADINGS
\usepackage{sectsty}
\usepackage[normalem]{ulem}
\sectionfont{\mdseries\upshape\Large}
\subsectionfont{\mdseries\scshape\normalsize}
\subsubsectionfont{\mdseries\upshape\large}

\renewenvironment{abstract}{%
{\mdseries\scshape\Large\abstractname}
\vspace{1em}\\
}{\par\noindent}

% LISTINGS
\usepackage{listings}
\usepackage{color}
\usepackage{appendix}

\usepackage{color}
\definecolor{codered}{rgb}{0.61,0.21,0.18}
\definecolor{codegreen}{rgb}{0,0.6,0}
\definecolor{codegray}{rgb}{0.5,0.5,0.5}
\definecolor{codepurple}{rgb}{0.58,0,0.82}
\definecolor{backcolour}{rgb}{1.0,1.0,1.0}
\lstset{
  backgroundcolor=\color{backcolour},   
  commentstyle=\color{codegray},
  keywordstyle=\color{codered},
  numberstyle=\tiny\color{codegreen},
  stringstyle=\color{codepurple},
  basicstyle=\footnotesize\ttfamily,        % the size of the fonts that are used for the code
  breaklines=true,                          % sets automatic line breaking
  keepspaces=true,                          % keeps spaces in text, useful for keeping indentation of code
  showspaces=false,                         % show spaces everywhere adding particular underscores; it overrides 'showstringspaces'
  showstringspaces=false,                   % underline spaces within strings only
  showtabs=false,                           % show tabs within strings adding particular underscores
  stepnumber=2,                             % the step between two line-numbers. If it's 1, each line will be numbered
  tabsize=2, 	                            % sets default tabsize to 2 spaces
  title=\lstname                            % show the filename of files included with \lstinputlisting
}


%     %%%%%%%%%%%%%%%
%
%     D O C U M E N T
%
%     %%%%%%%%%%%%%%%


\begin{document}
\title{IAR Task 2 Report}
\author{Angus Pearson -- s1311631\\ Jevgenij Zubovskij -- s1346981}
\date{\today}
\maketitle

%       ^v^v^v^v^v^v^v^v^v^v^v^v^v^v^v^v^v^v^v^v^v^v^v^v^v^v^v^v^v^v^v^v^v^v^v^


\begin{abstract}
  We present an implementation of an algorithm similar to \textit{Bug2}\cite{principlesrobot} 
  on top of reactive obstacle collision avoidance and edge following developed for \textit{Task1}.
  The existing architecture of storing timestamped poses in a Redis\cite{Redis} key/value 
  \& PubSub server is extended to include Goal state publishing and real-time visualisation 
  using a Redis to ROS\cite{ROS} pipe, enabling odometry, sensory information and goals
  to be displayed in Rviz. After-the-fact plotting is provided with MatPlotLib independent
  of ROS.
  
  Ideally the robot will be able to navigate to within 10cm of the origin (it's start point) from 
  an abitary location in multiple environments, including in a dynamic environment, say where other 
  actors (e.g. Humans) present a transient obstacle. In our testing, the robot did this successfully
  in ${12/15}$ tests, where the environment was designed to evoke edge-case behaviour considered
  hard for the algorithm.
\end{abstract}

%       ^v^v^v^v^v^v^v^v^v^v^v^v^v^v^v^v^v^v^v^v^v^v^v^v^v^v^v^v^v^v^v^v^v^v^v^


\section{Introduction}

The second IAR assignment entails extending the existing systems from Task 1 with Odometry, 
to maintain an estimate of the robot's location, enabling rendering of it's movement both 
in real time and after completion. The task also calls for `Return to home' ability, either 
``by retracing its outward route or more directly''. As an extention, basic mapping (without
filtering to correct for odometry dead-reckoning drift) of the world is desireable. We utilise 
Rviz, a popular graphical tool for ROS to live-render the Khepera Robot's pose, odometry trail,
obstacles perceived by the IR range sensors and goal (a plan-line to the origin).



%       ^v^v^v^v^v^v^v^v^v^v^v^v^v^v^v^v^v^v^v^v^v^v^v^v^v^v^v^v^v^v^v^v^v^v^v^


\section{Sect}

\subsection{Stuff}




\subsection{Force-Vector Control}



%       ^v^v^v^v^v^v^v^v^v^v^v^v^v^v^v^v^v^v^v^v^v^v^v^v^v^v^v^v^v^v^v^v^v^v^v^


\section{Developing Object-Following Behaviour}



\subsection{Sensor Use}


\subsection{Boredom, a useful concept}


\section{Algorithm}




%       ^v^v^v^v^v^v^v^v^v^v^v^v^v^v^v^v^v^v^v^v^v^v^v^v^v^v^v^v^v^v^v^v^v^v^v^

\newpage
\section{Results}


\section{Discussion \& Possible Improvements}


%       ^v^v^v^v^v^v^v^v^v^v^v^v^v^v^v^v^v^v^v^v^v^v^v^v^v^v^v^v^v^v^v^v^v^v^v^


\begin{thebibliography}{2}

\bibitem{principlesrobot}
\par{Principles of Robot Motion: Theory, Algorithms, and Implementation}\\
\textit{Howie Choset}

\bibitem{Redis}
\par{Redis is an open source (BSD licensed), in-memory data structure store, used as database, cache and message broker. It supports data structures such as strings, hashes, lists, sets, sorted sets with range queries, bitmaps, hyperloglogs and geospatial indexes with radius queries.}\\
\textit{http://redis.io}

\bibitem{ROS}
\par{The Robot Operating System (ROS) is a flexible framework for writing robot software. It is a collection of tools, libraries, and conventions that aim to simplify the task of creating complex and robust robot behavior across a wide variety of robotic platforms.}\\
\textit{http://www.ros.org/}

\end{thebibliography}


\begin{appendices}
\section*{Appendix}
\subsection{Code Listings}
\lstinputlisting[language=python]{../../main.py}
\lstinputlisting[language=python]{../../data.py}
\lstinputlisting[language=python]{../../state.py}
\lstinputlisting[language=python]{../../bug_state.py}
\lstinputlisting[language=python]{../../bug_algorithm.py}
\lstinputlisting[language=python]{../../navigation_algorithm.py}
\lstinputlisting[language=python]{../../odometry_algorithm.py}
\lstinputlisting[language=python]{../../odometry_state.py}
\lstinputlisting[language=python]{../../constants.py}
\end{appendices}


\end{document}
